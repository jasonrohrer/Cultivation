\documentclass[12pt]{article}
\usepackage{fullpage,times}


\makeatletter
\newcommand\TitleSize{\@setfontsize\TitleSize{100}{100}}

\newcommand\WebAddressSize{\@setfontsize\WebAddressSize{40}{40}}

\newcommand\BylineSize{\@setfontsize\BylineSize{20}{20}}

\newcommand\SectionHeadingSize{\@setfontsize\SectionHeadingSize{30}{30}}

\newcommand\BodySize{\@setfontsize\BodySize{14}{14}}

\makeatother

\pagestyle{empty}

%\textwidth 7.5in
\textheight 9.5in
%\oddsidemargin -0.5in

\begin{document}

\begin{center}
{\TitleSize Cultivation}\\
\vspace{0.25in}
{\BylineSize a video game by Jason Rohrer}
\end{center}


%\noindent {\SectionHeadingSize Conflict vs. compromise}

{\BodySize
Cultivation explores the social interactions within a gardening community.  You lead one family of gardeners, starting with a single individual, and wise choices can keep your genetic line from extinction.  While breeding plants, eating, and mating, your actions impact your neighbors, and the social balance sways between conflict and compromise.  In Cultivation, there is no shooting, but there are plenty of angry looks.
}

\vfill

\noindent {\SectionHeadingSize Game mechanics as metaphor }

\vspace{0.05in}

{\BodySize
One part of Cultivation's thesis might be summarized as follows:  {\it Conflict can cost both sides more than either stands to gain by winning.}  

In a novel or film, we could state this thesis outright or tell a story that demonstrates our point (the novel and film {\it The House of Sand and Fog} comes to mind as a work with a similar thesis).  In a game, we could also issue a bald thesis statement or tell an a story with a moral, but we have one additional tool for communication:  the mechanics of gameplay themselves.  We can design mechanics that lead the player down a particular intellectual path.

There is no right way to play Cultivation, just as there is no right way to interpret the series of events that emerge from a given play through the game.  This subtlety actually separates Cultivation somewhat from the Serious Games movement:  Cultivation is not just a ``game with a message,'' but a game that demands interpretation from the player.
}

\vfill

\noindent {\SectionHeadingSize Procedurally-generated content}

\vspace{0.05in}

{\BodySize
All of the graphics, sounds, melodies, and other content in Cultivation are 100\% procedurally generated at playtime.  In other words, there are no hand-painted texture maps---instead, each object has a uniquely ``grown'' appearance.  Every time you play, Cultivation generates fresh visuals, music, and behaviors.  Even though Cultivation is a game with a virtually limitless pallete of visual variety, it still fits easily on a single floppy disk.
}

\vfill

\begin{center}
{\BylineSize Free download for Windows, Mac, and Linux:}\\
\vspace{0.125in}
{\WebAddressSize http://cultivation.sf.net}
\end{center}

\end{document}